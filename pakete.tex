
% sysconfig und pakete
\usepackage[ngerman]{babel}
\usepackage[utf8]{inputenc}
\usepackage{graphicx}                   	%
\usepackage{tikz}					        % tikz-grafiken
\usetikzlibrary{shapes.multipart,trees,arrows,shapes,fit,backgrounds,topaths,positioning,fadings,decorations,automata}
\usepackage{pgfbaselayers}                  % GFX-Layer
\usepackage{eurosym}                    	% \euro Währungszeichen
\usepackage{geometry}                   	% Einfache Definition der Zeilenabstände, Seitenränder etc.
\usepackage{setspace}
\usepackage{todonotes}                  	% \todo{} Anmerkungen
\usepackage[binary]{SIunits}			% \mega\byte -- http://www.ctan.org/tex-archive/macros/latex/contrib/SIunits/
\usepackage{varioref}			% automatische Seiten-referenz etc.
\usepackage{url}
\usepackage{appendix}
\usepackage{paralist}           % Inline-Listen
\usepackage{tabularx}           % variabel/feste Tabellen-Spalten
\usepackage{booktabs}			% tabular-stuff

%% header/footer
\usepackage[nouppercase,                % nicht komplett in Großbuchstaben
    automark,                           % Kapitelangaben in Kopfzeile automatisch erstellen
    headsepline,                        % Trennlinie unter Kopfzeile
    ilines                              % Trennlinie linksbündig ausrichten
    ]{scrpage2}


\usepackage[acronym,			% new glossary with acronyms
    toc,				% add to TOC
    numberline,         % add page-number in TOC
    %style=listdotted,			% glossaries.pdf, page 55
    style=list,			% glossaries.pdf, page 55
    %header=plain,
    %cols=3,
    footnote,				% create footnote on new-acronym
    %toctitle={},
    numberedsection,
    section=chapter
]{glossaries}        % Glossar und Abkürzungen
\usepackage{makeidx}                    % Index-Ausgabe mit \printindex

% Symbolverzeichnis ------------------------------------------------------------
%   Symbolverzeichnisse bequem erstellen. Beruht auf MakeIndex:
%     makeindex.exe %Name%.nlo -s nomencl.ist -o %Name%.nls
%   erzeugt dann das Verzeichnis. Dieser Befehl kann z.B. im TeXnicCenter
%   als Postprozessor eingetragen werden, damit er nicht st�ndig manuell
%   ausgef�hrt werden muss.
%   Die Definitionen sind ausgegliedert in die Datei "Glossar.tex".
% ------------------------------------------------------------------------------
\usepackage[intoc]{nomencl}
\let\abbrev\nomenclature
\renewcommand{\nomname}{Abkürzungsverzeichnis}
\setlength{\nomlabelwidth}{.25\hsize}
\renewcommand{\nomlabel}[1]{#1 \dotfill}
\setlength{\nomitemsep}{-\parsep}

% Zum Einbinden von Quellcode
\usepackage{listings}
\usepackage{xcolor}
\definecolor{hellgelb}{rgb}{1,1,0.9}
\definecolor{colKeys}{rgb}{0,0,1}
\definecolor{colIdentifier}{rgb}{0,0,0}
\definecolor{colComments}{rgb}{1,0,0}
\definecolor{colString}{rgb}{0,0.5,0}
\lstset{%
    float=hbp,%
    basicstyle=\texttt\small, %
    identifierstyle=\color{colIdentifier}, %
    keywordstyle=\color{colKeys}, %
    stringstyle=\color{colString}, %
    commentstyle=\color{colComments}, %
    columns=flexible, %
    tabsize=2, %
    frame=single, %
    extendedchars=true, %
    showspaces=false, %
    showstringspaces=false, %
    numbers=left, %
    numberstyle=\tiny, %
    breaklines=true, %
    backgroundcolor=\color{hellgelb}, %
    breakautoindent=true, %
    escapeinside={(*@}{@*)}% use (*@\label{comment}@*)
%    captionpos=b%
}



% Wichtig für korrekte Zitierweise, use bibtex-style dinat
\usepackage{natbib}
\usepackage[chapter,numbib,numindex,notlof]{tocbibind}
% Quellenangaben in eckige Klammern fassen
%\bibpunct{[}{]}{;}{a}{}{,~}


\usepackage[colorlinks,			% Links farbig hervorheben
%    breaklinks,					% Links umbrechen, passiert automatisch bei nutzung von pdftex
    bookmarks,					%
    hyperindex=false,
    % uncomment the following line if you want to have black links (e.g., for printing)
    urlcolor=black, linkcolor=black, citecolor=black, %pagecolor=Black,%
    pdfauthor={Oluf Lorenzen},
    pdftitle={Split-Access-Routing mit Priorisierung auf Linux-Basis},
    pdfkeywords={Linux,Debian,routing}]{hyperref}
                                        % PDF-Gefrickel
                                        % Lange URLs umbrechen ...
%\usepackage{breakurl}

\usepackage{courier}

% debugging etc.
\usepackage[l2tabu, orthodox, abort]{nag}
